
\section{Введение}
\begin{frame}{\insertsectionhead}

\begin{columns}[onlytextwidth,T]
  \begin{column}{.55\linewidth}
    \small
    \begin{itemize}
        \item Саратовская область - субъект Российской федерации, входит в состав Приволжского федерального округа
        \item Саратовская область расположена в европейской части Российской Федерации, в юго-восточной части 
        Восточно-Европейской равнины на территории Нижнего Поволжья
        \item Административный центр - Саратов
        \item Основными сферами хозяйственной деятельности в регионе являются сельское хозяйство, химическая и нефтехимическая промышленность
    \end{itemize}
  \end{column}

  \begin{column}{.43\linewidth}
    \begin{center}
        \includegraphics[width=.2\textwidth]{assets/logo.png}
    \end{center}
    \includegraphics[width=\textwidth]{assets/saratov.png}
  \end{column}

\end{columns}
\end{frame}



\section{Общая характеристика состояния атмосферы}
\begin{frame}{\insertsectionhead}
    \begin{columns}
        \begin{column}{.55\linewidth}
        \footnotesize
        Выбросы загрязняющих веществ от стационарных и 
        передвижных источников на территории Саратовской области в 2018 году составили 377,2 тыс. т.
        По сравнению с предыдущим годом выбросы от стационарных источников уменьшились на 4,6 тыс. т (на 3,8\%), а 
        выбросы от автотранспорта увеличились на 10,4 тыс. т (на 4,2\%).
        
        ~ % Invisible

        На протяжении последних лет основными вкладчиками в промышленные выбросы области являются предприятия транспортировки и хранения и предприятия обрабатывающего производства. 
        \end{column}

        \begin{column}{.42\linewidth}
            \includegraphics[width=1.1\textwidth]{assets/tes.jpg}
        \end{column}
    \end{columns}
\end{frame}

\begin{frame}{\insertsectionhead}
    \begin{columns}
        \begin{column}{.55\linewidth}
        \footnotesize
        Увеличение выбросов загрязняющих веществ в атмосферу от 
        автотранспорта произошло во всех крупных городах области, что связано с увеличением количества зарегистрированных автотранспортных средств.
        
        ~ % Invisible

        Наибольшее загрязнение атмосферного воздуха регистрируется на улицах с 
        интенсивным движением в точках измерений крупных
        городов области: Саратов, Балаково, Балашов.
        \end{column}

        \begin{column}{.42\linewidth}
            \includegraphics[width=1.1\textwidth]{assets/vihlopi.jpg}
        \end{column}
    \end{columns}
\end{frame}


\section{Показатели проб относительно ПДК}
\begin{frame}{\insertsectionhead}
\end{frame}

\section{Главный фактор загрязнения воздуха}
\begin{frame}{\insertsectionhead}
\end{frame}

\section{Динамика выбросов загрязняющих веществ}
\begin{frame}{\insertsectionhead}
\end{frame}

\section{Структура выбросов загрязняющих веществ \\ от
стационарных источников}
\begin{frame}{\insertsectionhead}
\end{frame}

\section{Мониторинг загрязнения воздуха}
\begin{frame}{\insertsectionhead}
\end{frame}

\section{Деятельность предприятий в области \\ охраны окружающей среды}
\begin{frame}{\insertsectionhead}
\end{frame}

\section{Воздействие выхлопных газов автомобилей}
\begin{frame}{\insertsectionhead}
\end{frame}


\section{Меры по защите атмосферы от загрязнений}
\begin{frame}{\insertsectionhead}
    \begin{itemize}
        \item Установка очистных сооружений на предприятиях
        \item Использование экологически чистых видов топлива (газ, биодизель)
        \item Использование безотходных технологий
        \item Создание зеленых насаждений вдоль магистралей
        \item Повышение контроля за соблюдением экологических норм 
            по выбросу токсичных веществ
    \end{itemize}
\end{frame}



\section{Источники информации}
\begin{frame}{\insertsectionhead}
\end{frame}
